\documentclass{beamer}
\usepackage[utf8]{inputenc}
\usepackage{amsmath}
\usepackage{graphicx}
\usepackage{booktabs}
\usepackage{hyperref}
\usepackage{CJKutf8}
\usepackage{indentfirst}
\usepackage{float}
\usepackage{ragged2e} % For better text alignment in columns

\definecolor{myteal}{cmyk}{0.5,0.2,0.15,0}
\usecolortheme[named=myteal]{structure}

% --- Beamer 主题和颜色方案 ---
\usetheme{Madrid}
\usecolortheme{default}

% --- 标题信息 ---
\title[Image Captioning]{From Pixels to Prose: An Architectural Analysis of Image Captioning}
\author{
  \begin{CJK}{UTF8}{gbsn}王子恒 \quad 娄毅彬 \quad 方酉城 \quad 芮煜涵\end{CJK}
}
\date{\today}
\institute{SUSTech/Computer Vision} % 您可以填写您的学校或机构名称

% --- Beamer 设置 ---
\setbeamertemplate{navigation symbols}{} % 隐藏导航按钮
\setbeameroption{show notes} % 如果您想添加备注,可以取消注释

\begin{document}
\begin{CJK}{UTF8}{gbsn}

% =================================
% --- 标题页 ---
% =================================
\begin{frame}
  \titlepage
\end{frame}

% =================================
% --- 目录 ---
% =================================
\begin{frame}{Outline}
  \tableofcontents
\end{frame}

% =================================
% --- 1. 引言 ---
% =================================
\section{Introduction}

\begin{frame}{Introduction: What is Image Captioning?}
  \begin{itemize}
    \item \textbf{核心任务:} 为给定图像生成流畅、准确的自然语言描述。
    \item \textbf{技术交叉点:} 计算机视觉 (CV) 与 自然语言处理 (NLP)。
    \item \textbf{广泛应用:}
      \begin{itemize}
        \item 无障碍访问 (Accessibility)
        \item 内容检索 (Content Retrieval)
        \item 人机交互 (Human-Computer Interaction)
      \end{itemize}
  \end{itemize}
  \vfill
  \begin{center}
      \includegraphics[width=0.7\textwidth]{images/project_logo.png}
  \end{center}
\end{frame}

\begin{frame}{Our Project: A Modular Framework}
    \begin{block}{核心思想}
        大多数现代模型遵循 \textbf{编码器-解码器} 范式,但组件选择缺乏系统性分析。
    \end{block}
    
    \begin{itemize}
        \item \textbf{目标:} 设计一个模块化的视觉-语言框架,以灵活组合和评估不同架构。
        \item \textbf{实验组件:}
        \begin{itemize}
            \item \textbf{视觉编码器:} ResNet, Vision Transformer (ViT), Vision Mamba (Vim)
            \item \textbf{连接器:} MLP, Q-Former
            \item \textbf{语言解码器:} GPT-2, Qwen2, LLaMA 3, Mamba
        \end{itemize}
        \item \textbf{评估:} 在 \textbf{COCO 数据集} 上,使用 \textbf{BLEU} 和 \textbf{CIDEr} 指标进行评测。
    \end{itemize}
    
    \vfill
    \begin{block}{项目代码}
        \url{https://github.com/EarendelH/CV-Project-Image-Captioning}
    \end{block}
\end{frame}

% =================================
% --- 2. 相关工作 ---
% =================================
\section{Related Work}

\begin{frame}{The Evolution of Image Captioning Models}
    \begin{columns}[T]
        \begin{column}{0.5\textwidth}
            \begin{block}{1. CNN-RNN 时代}
                \textbf{模型:} Show and Tell \\
                \textbf{架构:} CNN 编码器 + RNN 解码器 \\
                \textbf{核心问题:} 信息瓶颈 (Information Bottleneck)
            \end{block}
            
            \begin{block}{2. 注意力机制革命}
                 \textbf{模型:} Show, Attend and Tell \\
                 \textbf{创新:} 引入视觉注意力机制,动态聚焦图像区域。\\
                 \textbf{优点:} 提升细节描述能力,提供可解释性。
            \end{block}
        \end{column}
        
        \begin{column}{0.5\textwidth}
            \begin{block}{3. Transformer 的崛起}
                \textbf{编码器:} Vision Transformer (ViT) \\
                \textbf{解码器:} Transformer Decoders (e.g., GPT-2) \\
                \textbf{优势:} 强大的全局上下文建模能力。
            \end{block}
            
            \begin{block}{4. 大视觉语言模型 (LVLMs)}
                 \textbf{范式:} 冻结的视觉编码器 + 冻结的LLM + 轻量级可训练连接器。\\
                 \textbf{代表:} LLaVA, BLIP-2
            \end{block}
        \end{column}
    \end{columns}
\end{frame}

% =================================
% --- 3. 方法/架构 ---
% =================================
\section{Approach: Core Components}

\begin{frame}{Component 1: Vision Encoders}
    \begin{itemize}
        \item \textbf{ResNet (CNN):}
        \begin{itemize}
            \item \textbf{优点:} 归纳偏置强 (Locality),数据高效。
            \item \textbf{缺点:} 难以捕捉全局关系。
        \end{itemize}
        \medskip
        
        \item \textbf{Vision Transformer (ViT):}
        \begin{itemize}
            \item \textbf{优点:} 自注意力机制,全局感受野强。
            \item \textbf{缺点:} 计算复杂度高 ($O(N^2)$)。
        \end{itemize}
        \medskip

        \item \textbf{Vision Mamba (Vim):}
        \begin{itemize}
            \item \textbf{优点:} 线性时间复杂度 ($O(N)$),高效处理高分辨率图像。
        \end{itemize}
        \medskip
        
        \item \textbf{MambaVision (Hybrid):}
        \begin{itemize}
             \item CVPR 2025 Nvidia 提出的模型,结合 Mamba 的效率和 Transformer 的自注意力,达到 SOTA 性能。
        \end{itemize}
    \end{itemize}
\end{frame}

\begin{frame}{Component 2: Language Decoders}
    \begin{itemize}
        \item \textbf{Autoregressive Transformers (GPT-2, LLaMA, Qwen):}
        \begin{itemize}
            \item \textbf{优点:} 在大规模文本上预训练,拥有丰富的语言和世界知识,生成质量高。
            \item \textbf{缺点:} 模型巨大,可能产生“幻觉”(Hallucination)。
        \end{itemize}
        \medskip
        
        \item \textbf{Mamba Language Model:}
        \begin{itemize}
            \item \textbf{优点:} 基于状态空间模型 (SSM),线性时间复杂度,处理长序列高效。
            \item \textbf{缺点:} 作为解码器应用仍在探索中。
        \end{itemize}
    \end{itemize}
    \vfill
    \begin{center}
        \includegraphics[width=0.38\textwidth]{images/mambavision_architecture.png}
    \end{center}
\end{frame}

\begin{frame}{Component 3: Vision-Language Connectors}
    \begin{block}{目标}
        将视觉特征“翻译”成大型语言模型 (LLM) 能够理解的格式。
    \end{block}
    
    \begin{columns}[T]
        \begin{column}{0.5\textwidth}
            \textbf{MLP Projector:}
            \begin{itemize}
                \item \textbf{方法:} 使用简单的多层感知机 (MLP) 投射特征。
                \item \textbf{优点:} 简单、高效。
                \item \textbf{缺点:} 可能将大量冗余的视觉令牌传递给LLM,增加计算负担。
            \end{itemize}
        \end{column}
        
        \begin{column}{0.5\textwidth}
            \textbf{Q-Former:}
            \begin{itemize}
                \item \textbf{方法:} 使用少量可学习的查询向量通过交叉注意力“审问”视觉特征。
                \item \textbf{优点:} 智能信息瓶颈,提取简洁摘要,提升性能和效率。
            \end{itemize}
            
        \end{column}
    \end{columns}
    \begin{center}
        \includegraphics[width=0.7\textwidth]{images/qformer_architecture.png}
    \end{center}
\end{frame}


% =================================
% --- 4. 实验结果 ---
% =================================
\section{Experimental Results}

\begin{frame}{Experiment Setup}
    \begin{itemize}
        \item \textbf{数据集:}
        \begin{itemize}
            \item \textbf{MS COCO2017:} 超过12万张图片,每张有5个人工标注的标题,是行业的黄金标准。
        \end{itemize}
        \medskip
        
        \item \textbf{评估指标:}
        \begin{itemize}
            \item \textbf{BLEU:} 衡量n-gram的精确度,但与人类判断相关性较弱。
            \item \textbf{CIDEr:} 专为字幕设计,通过TF-IDF加权的n-gram相似度衡量共识,与人类判断相关性好。
        \end{itemize}
    \end{itemize}
\end{frame}

\begin{frame}{Example}
    \begin{figure}
        \centering
        \includegraphics[width=0.4\linewidth]{images/example_image.jpg}
        \caption{Example image}
        \label{fig:enter-label}
    \end{figure}

    \begin{description}
        \item[GT1] An office desk with a row of books and a desktop computer.
        \item[GT2] A desktop computer next to a laptop computer.
        \item[GT3] A desk with a laptop, some keyboards, and some books.
        \item[GT4] There is a desktop computer and a laptop sitting on an organized desk in an office.
        \item[GT5] An office desk with a monitor, laptop, and books.
        \item[Generated] Computer desk with a monitor and keyboard on it.
    \end{description}
    
\end{frame}

% ------
\begin{frame}[fragile]{Ablation 1: Vision Encoders}
    \frametitle{Ablation Study of Vision Encoders (Decoder: GPT-2, Connector: MLP)}
    \resizebox{\textwidth}{!}{%
    \begin{tabular}{lcccccc}
    \toprule
    \textbf{Model (Vision Encoder)} & \textbf{Architecture} & \textbf{BLEU-1} & \textbf{BLEU-2} & \textbf{BLEU-3} & \textbf{BLEU-4} & \textbf{CIDEr} \\
    \midrule
    ResNet-50 & CNN & 59.3 & 44.2 & 26.1 & 20.3 & 90.4 \\
    vit-base-patch16 & Transformer & 63.7 & 51.1 & 33.6 & 24.3 & 98.8 \\
    Vision Mamba vim\_tiny & Mamba & 64.7 & 48.7 & 32.4 & 26.2 & 97.1 \\
    MambaVision & Hybrid Mamba-Transformer & \textbf{68.5} & \textbf{53.2} & \textbf{37.1} & \textbf{28.9} & \textbf{104.2} \\
    \bottomrule
    \end{tabular}
    }
    \vfill
    \begin{block}{Takeaway}
        基于 Transformer 和 Mamba 的架构因其全局上下文建模能力而优于 CNN。混合架构表现最佳。
    \end{block}
\end{frame}

\begin{frame}[fragile]{Ablation 2: Language Decoders}
    \frametitle{Ablation Study of Language Decoders (Encoder: ViT-Base, Connector: MLP)}
    \resizebox{\textwidth}{!}{%
    \begin{tabular}{lcccccc}
    \toprule
    \textbf{Model (Language Decoder)} & \textbf{Architecture} & \textbf{BLEU-1} & \textbf{BLEU-2} & \textbf{BLEU-3} & \textbf{BLEU-4} & \textbf{CIDEr} \\
    \midrule
    GPT-2 & Transformer & 63.7 & 51.1 & 33.6 & 24.3 & 98.8 \\
    QWen 3B & Transformer & 65.3 & 53.5 & 34.8 & 26.9 & 105.4 \\
    LLaMA 3B & Transformer & 66.4 & 52.4 & 34.7 & 27.1 & 108.6 \\
    Mamba-130m & Mamba & \textbf{69.2} & \textbf{54.3} & \textbf{37.1} & \textbf{29.2} & \textbf{110.7} \\
    \bottomrule
    \end{tabular}
    }
    \vfill
    \begin{block}{Takeaway}
        语言解码器的能力是性能的关键驱动因素。更强大的基础模型(如LLaMA, Mamba)能带来显著的性能提升。
    \end{block}
\end{frame}

\begin{frame}[fragile]{Ablation 3: Vision-Language Connectors}
    \frametitle{Ablation Study of Connectors (Encoder: ViT-Base, Decoder: GPT-2)}
    \resizebox{\textwidth}{!}{%
    \begin{tabular}{lccccccc}
    \toprule
    \textbf{Model (Connector)} & \textbf{Architecture} & \textbf{BLEU-1} & \textbf{BLEU-2} & \textbf{BLEU-3} & \textbf{BLEU-4} & \textbf{CIDEr} & \textbf{Time per epoch} \\
    \midrule
    MLP & 2-hidden layer & 63.7 & 51.1 & 33.6 & 24.3 & 98.8 & $\sim$8h \\
    Q-Former & Transformer & \textbf{68.8} & \textbf{52.7} & \textbf{38.3} & \textbf{28.4} & \textbf{100.4} & $\sim$4h \\
    \bottomrule
    \end{tabular}
    }
    \vfill
    \begin{block}{Takeaway}
        Q-Former 通过其过滤和冻结机制,不仅提升了CIDEr分数,还减少了训练时间,证明了更简洁、清晰的视觉信号的对于LLM的有效性。
    \end{block}
\end{frame}


% =================================
% --- 5. Mamba vs Transformer ---
% =================================
\section{Bonus: Mamba vs Transformer}

\begin{frame}{Computational Complexity: Linear vs. Quadratic}
    \frametitle{Inference Time vs. Sequence Length}
    \begin{figure}[H]
        \centering
        \frame{\includegraphics[width=0.8\textwidth]{images/performance_benchmark.png}}
    \end{figure}
    
    \begin{itemize}
        \item \textbf{Transformer (GPT-2):} 曲线呈 \textbf{二次方} ($O(N^2)$) 增长。序列越长,时间消耗急剧增加。
        \item \textbf{Mamba:} 曲线呈 \textbf{线性} ($O(N)$) 增长。时间消耗与序列长度成正比,稳定可预测。
    \end{itemize}
\end{frame}

\begin{frame}{Why the Difference? Core Mechanisms}
    \begin{columns}[T]
        \begin{column}{0.5\textwidth}
            \begin{alertblock}{Transformer: Self-Attention}
                \begin{itemize}
                    \item \textbf{工作原理:} 每个词元(token)都需要与序列中\textbf{所有其他}词元进行比较,计算注意力分数。
                    \item \textbf{复杂度:} "All-to-All" 的比较导致 $O(N^2)$ 的复杂度。
                \end{itemize}
            \end{alertblock}
        \end{column}
        
        \begin{column}{0.5\textwidth}
            \begin{exampleblock}{Mamba: State Space Model (SSM)}
                \begin{itemize}
                    \item \textbf{工作原理:} 按顺序处理词元,维护一个压缩的“状态”(记忆)。只需关注\textbf{当前词元和内部状态}。
                    \item \textbf{复杂度:} 线性扫描 (Linear Scan) 导致 $O(N)$ 的复杂度。
                \end{itemize}
            \end{exampleblock}
        \end{column}
    \end{columns}
\end{frame}

% =================================
% --- 6. 结论与未来工作 ---
% =================================
\section{Conclusion \& Future Work}

\begin{frame}{Conclusion}
    \begin{itemize}
        \item \textbf{架构演进:} Image Captioning 已从简单的 CNN-RNN 发展到复杂的大型视觉语言模型 (LVLM) 范式。
        \medskip
        \item \textbf{关键性能驱动因素:}
        \begin{enumerate}
            \item \textbf{视觉编码器:} ViT 和 Mamba 的全局上下文建模至关重要。
            \item \textbf{语言解码器:} 更强大的 LLM (如 LLaMA) 显著提升字幕质量。
            \item \textbf{连接器:} 智能连接器 (如 Q-Former) 能有效提炼信息,提高效率和性能。
        \end{enumerate}
        \medskip
        \item \textbf{挑战:}
        \begin{itemize}
            \item 现有评估指标的局限性。
            \item 高昂的训练和部署成本。
        \end{itemize}
    \end{itemize}
\end{frame}

\begin{frame}{Future Work}
    \begin{itemize}
        \item \textbf{检索增强生成 (RAG):}
        \begin{itemize}
            \item 引入外部知识,提高事实准确性,减少幻觉。
        \end{itemize}
        \medskip
        
        \item \textbf{参数高效微调 (PEFT):}
        \begin{itemize}
            \item 使用 LoRA 等技术,降低大模型的微调成本。
        \end{itemize}
        \medskip
        
        \item \textbf{向视频扩展:}
        \begin{itemize}
            \item 将模型能力从静态图像扩展到动态视频数据。
        \end{itemize}
        \medskip
        
        \item \textbf{模型分析与可信赖 AI:}
        \begin{itemize}
            \item 开发更鲁棒、可解释、值得信赖的系统。
        \end{itemize}
    \end{itemize}
\end{frame}

% =================================
% --- 提问环节 ---
% =================================
\begin{frame}
  \begin{center}
    \Huge \textbf{Thank You!}
    \\[2em]
    \Large Questions?
  \end{center}
\end{frame}


\end{CJK}
\end{document}